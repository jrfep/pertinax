\documentclass[]{article}
\usepackage{lmodern}
\usepackage{amssymb,amsmath}
\usepackage{ifxetex,ifluatex}
\usepackage{fixltx2e} % provides \textsubscript
\ifnum 0\ifxetex 1\fi\ifluatex 1\fi=0 % if pdftex
  \usepackage[T1]{fontenc}
  \usepackage[utf8]{inputenc}
\else % if luatex or xelatex
  \ifxetex
    \usepackage{mathspec}
  \else
    \usepackage{fontspec}
  \fi
  \defaultfontfeatures{Ligatures=TeX,Scale=MatchLowercase}
\fi
% use upquote if available, for straight quotes in verbatim environments
\IfFileExists{upquote.sty}{\usepackage{upquote}}{}
% use microtype if available
\IfFileExists{microtype.sty}{%
\usepackage{microtype}
\UseMicrotypeSet[protrusion]{basicmath} % disable protrusion for tt fonts
}{}
\usepackage[margin=1in]{geometry}
\usepackage{hyperref}
\hypersetup{unicode=true,
            pdftitle={Análisis de las llamadas de A. pertinax (variable Tcall dur)},
            pdfauthor={José R. Ferrer Paris},
            pdfborder={0 0 0},
            breaklinks=true}
\urlstyle{same}  % don't use monospace font for urls
\usepackage{color}
\usepackage{fancyvrb}
\newcommand{\VerbBar}{|}
\newcommand{\VERB}{\Verb[commandchars=\\\{\}]}
\DefineVerbatimEnvironment{Highlighting}{Verbatim}{commandchars=\\\{\}}
% Add ',fontsize=\small' for more characters per line
\usepackage{framed}
\definecolor{shadecolor}{RGB}{248,248,248}
\newenvironment{Shaded}{\begin{snugshade}}{\end{snugshade}}
\newcommand{\AlertTok}[1]{\textcolor[rgb]{0.94,0.16,0.16}{#1}}
\newcommand{\AnnotationTok}[1]{\textcolor[rgb]{0.56,0.35,0.01}{\textbf{\textit{#1}}}}
\newcommand{\AttributeTok}[1]{\textcolor[rgb]{0.77,0.63,0.00}{#1}}
\newcommand{\BaseNTok}[1]{\textcolor[rgb]{0.00,0.00,0.81}{#1}}
\newcommand{\BuiltInTok}[1]{#1}
\newcommand{\CharTok}[1]{\textcolor[rgb]{0.31,0.60,0.02}{#1}}
\newcommand{\CommentTok}[1]{\textcolor[rgb]{0.56,0.35,0.01}{\textit{#1}}}
\newcommand{\CommentVarTok}[1]{\textcolor[rgb]{0.56,0.35,0.01}{\textbf{\textit{#1}}}}
\newcommand{\ConstantTok}[1]{\textcolor[rgb]{0.00,0.00,0.00}{#1}}
\newcommand{\ControlFlowTok}[1]{\textcolor[rgb]{0.13,0.29,0.53}{\textbf{#1}}}
\newcommand{\DataTypeTok}[1]{\textcolor[rgb]{0.13,0.29,0.53}{#1}}
\newcommand{\DecValTok}[1]{\textcolor[rgb]{0.00,0.00,0.81}{#1}}
\newcommand{\DocumentationTok}[1]{\textcolor[rgb]{0.56,0.35,0.01}{\textbf{\textit{#1}}}}
\newcommand{\ErrorTok}[1]{\textcolor[rgb]{0.64,0.00,0.00}{\textbf{#1}}}
\newcommand{\ExtensionTok}[1]{#1}
\newcommand{\FloatTok}[1]{\textcolor[rgb]{0.00,0.00,0.81}{#1}}
\newcommand{\FunctionTok}[1]{\textcolor[rgb]{0.00,0.00,0.00}{#1}}
\newcommand{\ImportTok}[1]{#1}
\newcommand{\InformationTok}[1]{\textcolor[rgb]{0.56,0.35,0.01}{\textbf{\textit{#1}}}}
\newcommand{\KeywordTok}[1]{\textcolor[rgb]{0.13,0.29,0.53}{\textbf{#1}}}
\newcommand{\NormalTok}[1]{#1}
\newcommand{\OperatorTok}[1]{\textcolor[rgb]{0.81,0.36,0.00}{\textbf{#1}}}
\newcommand{\OtherTok}[1]{\textcolor[rgb]{0.56,0.35,0.01}{#1}}
\newcommand{\PreprocessorTok}[1]{\textcolor[rgb]{0.56,0.35,0.01}{\textit{#1}}}
\newcommand{\RegionMarkerTok}[1]{#1}
\newcommand{\SpecialCharTok}[1]{\textcolor[rgb]{0.00,0.00,0.00}{#1}}
\newcommand{\SpecialStringTok}[1]{\textcolor[rgb]{0.31,0.60,0.02}{#1}}
\newcommand{\StringTok}[1]{\textcolor[rgb]{0.31,0.60,0.02}{#1}}
\newcommand{\VariableTok}[1]{\textcolor[rgb]{0.00,0.00,0.00}{#1}}
\newcommand{\VerbatimStringTok}[1]{\textcolor[rgb]{0.31,0.60,0.02}{#1}}
\newcommand{\WarningTok}[1]{\textcolor[rgb]{0.56,0.35,0.01}{\textbf{\textit{#1}}}}
\usepackage{longtable,booktabs}
\usepackage{graphicx}
% grffile has become a legacy package: https://ctan.org/pkg/grffile
\IfFileExists{grffile.sty}{%
\usepackage{grffile}
}{}
\makeatletter
\def\maxwidth{\ifdim\Gin@nat@width>\linewidth\linewidth\else\Gin@nat@width\fi}
\def\maxheight{\ifdim\Gin@nat@height>\textheight\textheight\else\Gin@nat@height\fi}
\makeatother
% Scale images if necessary, so that they will not overflow the page
% margins by default, and it is still possible to overwrite the defaults
% using explicit options in \includegraphics[width, height, ...]{}
\setkeys{Gin}{width=\maxwidth,height=\maxheight,keepaspectratio}
\IfFileExists{parskip.sty}{%
\usepackage{parskip}
}{% else
\setlength{\parindent}{0pt}
\setlength{\parskip}{6pt plus 2pt minus 1pt}
}
\setlength{\emergencystretch}{3em}  % prevent overfull lines
\providecommand{\tightlist}{%
  \setlength{\itemsep}{0pt}\setlength{\parskip}{0pt}}
\setcounter{secnumdepth}{0}
% Redefines (sub)paragraphs to behave more like sections
\ifx\paragraph\undefined\else
\let\oldparagraph\paragraph
\renewcommand{\paragraph}[1]{\oldparagraph{#1}\mbox{}}
\fi
\ifx\subparagraph\undefined\else
\let\oldsubparagraph\subparagraph
\renewcommand{\subparagraph}[1]{\oldsubparagraph{#1}\mbox{}}
\fi

%%% Use protect on footnotes to avoid problems with footnotes in titles
\let\rmarkdownfootnote\footnote%
\def\footnote{\protect\rmarkdownfootnote}

%%% Change title format to be more compact
\usepackage{titling}

% Create subtitle command for use in maketitle
\providecommand{\subtitle}[1]{
  \posttitle{
    \begin{center}\large#1\end{center}
    }
}

\setlength{\droptitle}{-2em}

  \title{Análisis de las llamadas de A. pertinax (variable Tcall dur)}
    \pretitle{\vspace{\droptitle}\centering\huge}
  \posttitle{\par}
    \author{José R. Ferrer Paris}
    \preauthor{\centering\large\emph}
  \postauthor{\par}
    \date{}
    \predate{}\postdate{}
  
\usepackage{graphicx}
\usepackage{float}

\begin{document}
\maketitle

Leemos el archivo de datos de Aratinga pertinax, versión de 2019, y
reclassificamos la variable Region para facilitar interpretación:

\begin{Shaded}
\begin{Highlighting}[]
\NormalTok{dts <-}\StringTok{ }\KeywordTok{read.csv}\NormalTok{(}\KeywordTok{sprintf}\NormalTok{(}\StringTok{"%s/data/mdf_JR_15viii19.csv"}\NormalTok{,script.dir))}

\KeywordTok{str}\NormalTok{(dts)}
\end{Highlighting}
\end{Shaded}

\begin{verbatim}
## 'data.frame':    1351 obs. of  9 variables:
##  $ IndivGroup: Factor w/ 97 levels "AUA01","AUA02",..: 1 1 1 1 1 1 1 1 2 2 ...
##  $ soundfile : Factor w/ 1351 levels "0211327a","0211344a",..: 1 2 3 4 5 6 7 8 9 10 ...
##  $ S1_dur    : num  0.156 0.133 0.14 0.136 0.149 ...
##  $ Tcall_dur : num  0.154 0.137 0.144 0.137 0.144 ...
##  $ RecSite   : Factor w/ 37 levels "A6","A7","B1",..: 1 1 1 1 1 1 1 1 1 1 ...
##  $ Lat       : num  12.5 12.5 12.5 12.5 12.5 ...
##  $ Long      : num  -69.9 -69.9 -69.9 -69.9 -69.9 ...
##  $ LocCode   : Factor w/ 14 levels "AUA","BON","CUR",..: 1 1 1 1 1 1 1 1 1 1 ...
##  $ Region    : Factor w/ 2 levels "isl","main": 1 1 1 1 1 1 1 1 1 1 ...
\end{verbatim}

\begin{Shaded}
\begin{Highlighting}[]
\NormalTok{dts}\OperatorTok{$}\NormalTok{Region <-}\StringTok{ }\KeywordTok{factor}\NormalTok{(dts}\OperatorTok{$}\NormalTok{Region,}\DataTypeTok{levels=}\KeywordTok{c}\NormalTok{(}\StringTok{"main"}\NormalTok{,}\StringTok{"isl"}\NormalTok{))}
\NormalTok{dts <-}\StringTok{ }\KeywordTok{subset}\NormalTok{(dts,}\OperatorTok{!}\NormalTok{LocCode }\OperatorTok\StringTok{ }\KeywordTok{c}\NormalTok{(}\StringTok{"VEN3"}\NormalTok{,}\StringTok{"VEN7"}\NormalTok{,}\StringTok{"VEN8"}\NormalTok{,}\StringTok{"VEN9"}\NormalTok{))}
\NormalTok{dts}\OperatorTok{$}\NormalTok{LocCode <-}\StringTok{ }\KeywordTok{droplevels}\NormalTok{(dts}\OperatorTok{$}\NormalTok{LocCode)}
\end{Highlighting}
\end{Shaded}

Vamos a comparar ocho modelos para la variable Tcall\_dur

\begin{longtable}[]{@{}llll@{}}
\toprule
MODELO & efecto fijo & efecto aleatorio &
heterocedasticidad\tabularnewline
\midrule
\endhead
f000 & isla & constante & sin\tabularnewline
f010 & isla & isla & sin\tabularnewline
f001 & isla & constante & isla\tabularnewline
f011 & isla & isla & isla\tabularnewline
f100 & isla+long & constante & sin\tabularnewline
f110 & isla+long & isla & sin\tabularnewline
f101 & isla+long & constante & isla\tabularnewline
f111 & isla+long & isla & isla\tabularnewline
\bottomrule
\end{longtable}

El modelo nulo con efecto fijo de la isla

\begin{Shaded}
\begin{Highlighting}[]
\NormalTok{f000 <-}\StringTok{ }\KeywordTok{lme}\NormalTok{(Tcall_dur}\OperatorTok{~}\NormalTok{Region,dts,}\DataTypeTok{random=}\OperatorTok{~}\DecValTok{1}\OperatorTok{|}\NormalTok{LocCode}\OperatorTok{/}\NormalTok{IndivGroup, }\DataTypeTok{method=}\StringTok{"ML"}\NormalTok{)}
\end{Highlighting}
\end{Shaded}

Nulo + efecto aleatorio de isla/continente

\begin{Shaded}
\begin{Highlighting}[]
\NormalTok{f010 <-}\StringTok{ }\KeywordTok{lme}\NormalTok{(Tcall_dur}\OperatorTok{~}\NormalTok{Region,dts,}
  \DataTypeTok{random=}\KeywordTok{list}\NormalTok{(}\DataTypeTok{LocCode=}\KeywordTok{pdDiag}\NormalTok{(}\OperatorTok{~}\NormalTok{Region),}\DataTypeTok{IndivGroup=}\KeywordTok{pdDiag}\NormalTok{(}\OperatorTok{~}\NormalTok{Region)), }\DataTypeTok{method=}\StringTok{"ML"}\NormalTok{)}
\end{Highlighting}
\end{Shaded}

Nulo + heterocedasticidad

\begin{Shaded}
\begin{Highlighting}[]
\NormalTok{f001 <-}\StringTok{ }\KeywordTok{lme}\NormalTok{(Tcall_dur}\OperatorTok{~}\NormalTok{Region,dts,}\DataTypeTok{random=}\OperatorTok{~}\DecValTok{1}\OperatorTok{|}\NormalTok{LocCode}\OperatorTok{/}\NormalTok{IndivGroup,}\DataTypeTok{weights=}\KeywordTok{varIdent}\NormalTok{(}\DataTypeTok{form=}\OperatorTok{~}\DecValTok{1}\OperatorTok{|}\NormalTok{Region), }\DataTypeTok{method=}\StringTok{"ML"}\NormalTok{)}
\end{Highlighting}
\end{Shaded}

Nulo + efecto aleatorio de isla/continente + heterocedasticidad

\begin{Shaded}
\begin{Highlighting}[]
\NormalTok{f011 <-}
\StringTok{ }\KeywordTok{lme}\NormalTok{(Tcall_dur}\OperatorTok{~}\NormalTok{Region,dts,}
   \DataTypeTok{random=}\KeywordTok{list}\NormalTok{(}\DataTypeTok{LocCode=}\KeywordTok{pdDiag}\NormalTok{(}\OperatorTok{~}\NormalTok{Region),}\DataTypeTok{IndivGroup=}\KeywordTok{pdDiag}\NormalTok{(}\OperatorTok{~}\NormalTok{Region)),}
   \DataTypeTok{weights=}\KeywordTok{varIdent}\NormalTok{(}\DataTypeTok{form=}\OperatorTok{~}\DecValTok{1}\OperatorTok{|}\NormalTok{Region), }\DataTypeTok{method=}\StringTok{"ML"}\NormalTok{)}
\end{Highlighting}
\end{Shaded}

Modelo alternativo con efecto fijo de la isla y longitud

\begin{Shaded}
\begin{Highlighting}[]
\NormalTok{f100 <-}\StringTok{ }\KeywordTok{lme}\NormalTok{(Tcall_dur}\OperatorTok{~}\NormalTok{Region}\OperatorTok{+}\NormalTok{Long,dts,}\DataTypeTok{random=}\OperatorTok{~}\DecValTok{1}\OperatorTok{|}\NormalTok{LocCode}\OperatorTok{/}\NormalTok{IndivGroup, }\DataTypeTok{method=}\StringTok{"ML"}\NormalTok{)}
\end{Highlighting}
\end{Shaded}

Alternativo + efecto aleatorio de isla/continente

\begin{Shaded}
\begin{Highlighting}[]
\NormalTok{f110 <-}\StringTok{ }\KeywordTok{lme}\NormalTok{(Tcall_dur}\OperatorTok{~}\NormalTok{Region}\OperatorTok{+}\NormalTok{Long,dts,}
     \DataTypeTok{random=}\KeywordTok{list}\NormalTok{(}\DataTypeTok{LocCode=}\KeywordTok{pdDiag}\NormalTok{(}\OperatorTok{~}\NormalTok{Region),}\DataTypeTok{IndivGroup=}\KeywordTok{pdDiag}\NormalTok{(}\OperatorTok{~}\NormalTok{Region)), }\DataTypeTok{method=}\StringTok{"ML"}\NormalTok{)}
\end{Highlighting}
\end{Shaded}

Alternativo + heterocedasticidad

\begin{Shaded}
\begin{Highlighting}[]
\NormalTok{f101 <-}\StringTok{ }\KeywordTok{lme}\NormalTok{(Tcall_dur}\OperatorTok{~}\NormalTok{Region}\OperatorTok{+}\NormalTok{Long,dts,}\DataTypeTok{random=}\OperatorTok{~}\DecValTok{1}\OperatorTok{|}\NormalTok{LocCode}\OperatorTok{/}\NormalTok{IndivGroup,}\DataTypeTok{weights=}\KeywordTok{varIdent}\NormalTok{(}\DataTypeTok{form=}\OperatorTok{~}\DecValTok{1}\OperatorTok{|}\NormalTok{Region), }\DataTypeTok{method=}\StringTok{"ML"}\NormalTok{)}
\end{Highlighting}
\end{Shaded}

Alternativo + efecto aleatorio de isla/continente + heterocedasticidad

\begin{Shaded}
\begin{Highlighting}[]
\NormalTok{f111 <-}
\StringTok{      }\KeywordTok{lme}\NormalTok{(Tcall_dur}\OperatorTok{~}\NormalTok{Region}\OperatorTok{+}\NormalTok{Long,dts,}
        \DataTypeTok{random=}\KeywordTok{list}\NormalTok{(}\DataTypeTok{LocCode=}\KeywordTok{pdDiag}\NormalTok{(}\OperatorTok{~}\NormalTok{Region),}\DataTypeTok{IndivGroup=}\KeywordTok{pdDiag}\NormalTok{(}\OperatorTok{~}\NormalTok{Region)),}
        \DataTypeTok{weights=}\KeywordTok{varIdent}\NormalTok{(}\DataTypeTok{form=}\OperatorTok{~}\DecValTok{1}\OperatorTok{|}\NormalTok{Region), }\DataTypeTok{method=}\StringTok{"ML"}\NormalTok{)}
\end{Highlighting}
\end{Shaded}

\hypertarget{resultados}{%
\subsection{Resultados}\label{resultados}}

Comparamos el AIC de los modelos ajustados

\begin{Shaded}
\begin{Highlighting}[]
\KeywordTok{anova}\NormalTok{(f000,f010,f001,f011,}
\NormalTok{  f100,f110,f101,f111)}
\end{Highlighting}
\end{Shaded}

\begin{verbatim}
##      Model df       AIC       BIC   logLik   Test   L.Ratio p-value
## f000     1  5 -2125.771 -2099.848 1067.886                         
## f010     2  7 -2124.056 -2087.764 1069.028 1 vs 2   2.28474  0.3191
## f001     3  6 -2236.157 -2205.049 1124.078 2 vs 3 110.10053  <.0001
## f011     4  8 -2233.201 -2191.724 1124.600 3 vs 4   1.04403  0.5933
## f100     5  6 -2124.581 -2093.473 1068.290 4 vs 5 112.61992  <.0001
## f110     6  8 -2123.759 -2082.282 1069.879 5 vs 6   3.17799  0.2041
## f101     7  7 -2235.237 -2198.944 1124.618 6 vs 7 109.47793  <.0001
## f111     8  9 -2232.900 -2186.238 1125.450 7 vs 8   1.66301  0.4354
\end{verbatim}

Reordenamos los modelos según el AIC

\begin{Shaded}
\begin{Highlighting}[]
\NormalTok{mis.aics <-}\StringTok{ }\KeywordTok{AIC}\NormalTok{(f000,f010,f001,f011,}
\NormalTok{  f100,f110,f101,f111)}
\NormalTok{aic.tab <-}\StringTok{ }\KeywordTok{cbind}\NormalTok{(mis.aics,}\DataTypeTok{delta.AIC=}\NormalTok{mis.aics[,}\DecValTok{2}\NormalTok{]}\OperatorTok{-}\KeywordTok{min}\NormalTok{(mis.aics[,}\DecValTok{2}\NormalTok{]))}
\NormalTok{aic.tab[}\KeywordTok{order}\NormalTok{(aic.tab}\OperatorTok{$}\NormalTok{AIC),]}
\end{Highlighting}
\end{Shaded}

\begin{verbatim}
##      df       AIC   delta.AIC
## f001  6 -2236.157   0.0000000
## f101  7 -2235.237   0.9199722
## f011  8 -2233.201   2.9559659
## f111  9 -2232.900   3.2569575
## f000  5 -2125.771 110.3852646
## f100  6 -2124.581 111.5758840
## f010  7 -2124.056 112.1005254
## f110  8 -2123.759 112.3978989
\end{verbatim}

El Mejor modelo incluye efectos fijos de Isla, con heterocedasticidad.
El modelo con longitud en el efecto fijo tiene un soporte similar (delta
AIC \textless{} 2).

Los detalles del modelo a continuación:

\begin{Shaded}
\begin{Highlighting}[]
\KeywordTok{summary}\NormalTok{(f001)}
\end{Highlighting}
\end{Shaded}

\begin{verbatim}
## Linear mixed-effects model fit by maximum likelihood
##  Data: dts 
##         AIC       BIC   logLik
##   -2236.157 -2205.049 1124.078
## 
## Random effects:
##  Formula: ~1 | LocCode
##          (Intercept)
## StdDev: 6.125629e-06
## 
##  Formula: ~1 | IndivGroup %in% LocCode
##         (Intercept)  Residual
## StdDev:  0.06162041 0.1194656
## 
## Variance function:
##  Structure: Different standard deviations per stratum
##  Formula: ~1 | Region 
##  Parameter estimates:
##      isl     main 
## 1.000000 0.645795 
## Fixed effects: Tcall_dur ~ Region 
##                  Value   Std.Error   DF   t-value p-value
## (Intercept) 0.24916704 0.009566891 1226 26.044724  0.0000
## Regionisl   0.05009161 0.014349363    8  3.490859  0.0082
##  Correlation: 
##           (Intr)
## Regionisl -0.667
## 
## Standardized Within-Group Residuals:
##         Min          Q1         Med          Q3         Max 
## -2.72942119 -0.50238016 -0.09720572  0.41615095  5.63287779 
## 
## Number of Observations: 1319
## Number of Groups: 
##                 LocCode IndivGroup %in% LocCode 
##                      10                      93
\end{verbatim}

\begin{Shaded}
\begin{Highlighting}[]
\KeywordTok{intervals}\NormalTok{(f001,}\DataTypeTok{which=}\StringTok{"fixed"}\NormalTok{)}
\end{Highlighting}
\end{Shaded}

\begin{verbatim}
## Approximate 95% confidence intervals
## 
##  Fixed effects:
##                  lower       est.     upper
## (Intercept) 0.23041199 0.24916704 0.2679221
## Regionisl   0.01702702 0.05009161 0.0831562
## attr(,"label")
## [1] "Fixed effects:"
\end{verbatim}

\begin{Shaded}
\begin{Highlighting}[]
\KeywordTok{VarCorr}\NormalTok{(f001)}
\end{Highlighting}
\end{Shaded}

\begin{verbatim}
##              Variance     StdDev      
## LocCode =    pdLogChol(1)             
## (Intercept)  3.752333e-11 6.125629e-06
## IndivGroup = pdLogChol(1)             
## (Intercept)  3.797075e-03 6.162041e-02
## Residual     1.427203e-02 1.194656e-01
\end{verbatim}


\end{document}
