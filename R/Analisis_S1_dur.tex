\documentclass[]{article}
\usepackage{lmodern}
\usepackage{amssymb,amsmath}
\usepackage{ifxetex,ifluatex}
\usepackage{fixltx2e} % provides \textsubscript
\ifnum 0\ifxetex 1\fi\ifluatex 1\fi=0 % if pdftex
  \usepackage[T1]{fontenc}
  \usepackage[utf8]{inputenc}
\else % if luatex or xelatex
  \ifxetex
    \usepackage{mathspec}
  \else
    \usepackage{fontspec}
  \fi
  \defaultfontfeatures{Ligatures=TeX,Scale=MatchLowercase}
\fi
% use upquote if available, for straight quotes in verbatim environments
\IfFileExists{upquote.sty}{\usepackage{upquote}}{}
% use microtype if available
\IfFileExists{microtype.sty}{%
\usepackage{microtype}
\UseMicrotypeSet[protrusion]{basicmath} % disable protrusion for tt fonts
}{}
\usepackage[margin=1in]{geometry}
\usepackage{hyperref}
\hypersetup{unicode=true,
            pdftitle={Análisis de las llamadas de A. pertinax (variable T1 dur)},
            pdfauthor={José R. Ferrer Paris},
            pdfborder={0 0 0},
            breaklinks=true}
\urlstyle{same}  % don't use monospace font for urls
\usepackage{color}
\usepackage{fancyvrb}
\newcommand{\VerbBar}{|}
\newcommand{\VERB}{\Verb[commandchars=\\\{\}]}
\DefineVerbatimEnvironment{Highlighting}{Verbatim}{commandchars=\\\{\}}
% Add ',fontsize=\small' for more characters per line
\usepackage{framed}
\definecolor{shadecolor}{RGB}{248,248,248}
\newenvironment{Shaded}{\begin{snugshade}}{\end{snugshade}}
\newcommand{\AlertTok}[1]{\textcolor[rgb]{0.94,0.16,0.16}{#1}}
\newcommand{\AnnotationTok}[1]{\textcolor[rgb]{0.56,0.35,0.01}{\textbf{\textit{#1}}}}
\newcommand{\AttributeTok}[1]{\textcolor[rgb]{0.77,0.63,0.00}{#1}}
\newcommand{\BaseNTok}[1]{\textcolor[rgb]{0.00,0.00,0.81}{#1}}
\newcommand{\BuiltInTok}[1]{#1}
\newcommand{\CharTok}[1]{\textcolor[rgb]{0.31,0.60,0.02}{#1}}
\newcommand{\CommentTok}[1]{\textcolor[rgb]{0.56,0.35,0.01}{\textit{#1}}}
\newcommand{\CommentVarTok}[1]{\textcolor[rgb]{0.56,0.35,0.01}{\textbf{\textit{#1}}}}
\newcommand{\ConstantTok}[1]{\textcolor[rgb]{0.00,0.00,0.00}{#1}}
\newcommand{\ControlFlowTok}[1]{\textcolor[rgb]{0.13,0.29,0.53}{\textbf{#1}}}
\newcommand{\DataTypeTok}[1]{\textcolor[rgb]{0.13,0.29,0.53}{#1}}
\newcommand{\DecValTok}[1]{\textcolor[rgb]{0.00,0.00,0.81}{#1}}
\newcommand{\DocumentationTok}[1]{\textcolor[rgb]{0.56,0.35,0.01}{\textbf{\textit{#1}}}}
\newcommand{\ErrorTok}[1]{\textcolor[rgb]{0.64,0.00,0.00}{\textbf{#1}}}
\newcommand{\ExtensionTok}[1]{#1}
\newcommand{\FloatTok}[1]{\textcolor[rgb]{0.00,0.00,0.81}{#1}}
\newcommand{\FunctionTok}[1]{\textcolor[rgb]{0.00,0.00,0.00}{#1}}
\newcommand{\ImportTok}[1]{#1}
\newcommand{\InformationTok}[1]{\textcolor[rgb]{0.56,0.35,0.01}{\textbf{\textit{#1}}}}
\newcommand{\KeywordTok}[1]{\textcolor[rgb]{0.13,0.29,0.53}{\textbf{#1}}}
\newcommand{\NormalTok}[1]{#1}
\newcommand{\OperatorTok}[1]{\textcolor[rgb]{0.81,0.36,0.00}{\textbf{#1}}}
\newcommand{\OtherTok}[1]{\textcolor[rgb]{0.56,0.35,0.01}{#1}}
\newcommand{\PreprocessorTok}[1]{\textcolor[rgb]{0.56,0.35,0.01}{\textit{#1}}}
\newcommand{\RegionMarkerTok}[1]{#1}
\newcommand{\SpecialCharTok}[1]{\textcolor[rgb]{0.00,0.00,0.00}{#1}}
\newcommand{\SpecialStringTok}[1]{\textcolor[rgb]{0.31,0.60,0.02}{#1}}
\newcommand{\StringTok}[1]{\textcolor[rgb]{0.31,0.60,0.02}{#1}}
\newcommand{\VariableTok}[1]{\textcolor[rgb]{0.00,0.00,0.00}{#1}}
\newcommand{\VerbatimStringTok}[1]{\textcolor[rgb]{0.31,0.60,0.02}{#1}}
\newcommand{\WarningTok}[1]{\textcolor[rgb]{0.56,0.35,0.01}{\textbf{\textit{#1}}}}
\usepackage{longtable,booktabs}
\usepackage{graphicx,grffile}
\makeatletter
\def\maxwidth{\ifdim\Gin@nat@width>\linewidth\linewidth\else\Gin@nat@width\fi}
\def\maxheight{\ifdim\Gin@nat@height>\textheight\textheight\else\Gin@nat@height\fi}
\makeatother
% Scale images if necessary, so that they will not overflow the page
% margins by default, and it is still possible to overwrite the defaults
% using explicit options in \includegraphics[width, height, ...]{}
\setkeys{Gin}{width=\maxwidth,height=\maxheight,keepaspectratio}
\IfFileExists{parskip.sty}{%
\usepackage{parskip}
}{% else
\setlength{\parindent}{0pt}
\setlength{\parskip}{6pt plus 2pt minus 1pt}
}
\setlength{\emergencystretch}{3em}  % prevent overfull lines
\providecommand{\tightlist}{%
  \setlength{\itemsep}{0pt}\setlength{\parskip}{0pt}}
\setcounter{secnumdepth}{0}
% Redefines (sub)paragraphs to behave more like sections
\ifx\paragraph\undefined\else
\let\oldparagraph\paragraph
\renewcommand{\paragraph}[1]{\oldparagraph{#1}\mbox{}}
\fi
\ifx\subparagraph\undefined\else
\let\oldsubparagraph\subparagraph
\renewcommand{\subparagraph}[1]{\oldsubparagraph{#1}\mbox{}}
\fi

%%% Use protect on footnotes to avoid problems with footnotes in titles
\let\rmarkdownfootnote\footnote%
\def\footnote{\protect\rmarkdownfootnote}

%%% Change title format to be more compact
\usepackage{titling}

% Create subtitle command for use in maketitle
\providecommand{\subtitle}[1]{
  \posttitle{
    \begin{center}\large#1\end{center}
    }
}

\setlength{\droptitle}{-2em}

  \title{Análisis de las llamadas de A. pertinax (variable T1 dur)}
    \pretitle{\vspace{\droptitle}\centering\huge}
  \posttitle{\par}
    \author{José R. Ferrer Paris}
    \preauthor{\centering\large\emph}
  \postauthor{\par}
    \date{}
    \predate{}\postdate{}
  
\usepackage{graphicx}
\usepackage{float}

\begin{document}
\maketitle

Leemos el archivo de datos de Aratinga pertinax, versión de 2019, y
reclassificamos la variable Region para facilitar interpretación:

\begin{Shaded}
\begin{Highlighting}[]
\NormalTok{dts <-}\StringTok{ }\KeywordTok{read.csv}\NormalTok{(}\KeywordTok{sprintf}\NormalTok{(}\StringTok{"%s/data/mdf_JR_15viii19.csv"}\NormalTok{,script.dir))}

\KeywordTok{str}\NormalTok{(dts)}
\end{Highlighting}
\end{Shaded}

\begin{verbatim}
## 'data.frame':    1351 obs. of  9 variables:
##  $ IndivGroup: Factor w/ 97 levels "AUA01","AUA02",..: 1 1 1 1 1 1 1 1 2 2 ...
##  $ soundfile : Factor w/ 1351 levels "0211327a","0211344a",..: 1 2 3 4 5 6 7 8 9 10 ...
##  $ S1_dur    : num  0.156 0.133 0.14 0.136 0.149 ...
##  $ Tcall_dur : num  0.154 0.137 0.144 0.137 0.144 ...
##  $ RecSite   : Factor w/ 37 levels "A6","A7","B1",..: 1 1 1 1 1 1 1 1 1 1 ...
##  $ Lat       : num  12.5 12.5 12.5 12.5 12.5 ...
##  $ Long      : num  -69.9 -69.9 -69.9 -69.9 -69.9 ...
##  $ LocCode   : Factor w/ 14 levels "AUA","BON","CUR",..: 1 1 1 1 1 1 1 1 1 1 ...
##  $ Region    : Factor w/ 2 levels "isl","main": 1 1 1 1 1 1 1 1 1 1 ...
\end{verbatim}

\begin{Shaded}
\begin{Highlighting}[]
\NormalTok{dts}\OperatorTok{$}\NormalTok{Region <-}\StringTok{ }\KeywordTok{factor}\NormalTok{(dts}\OperatorTok{$}\NormalTok{Region,}\DataTypeTok{levels=}\KeywordTok{c}\NormalTok{(}\StringTok{"main"}\NormalTok{,}\StringTok{"isl"}\NormalTok{))}
\end{Highlighting}
\end{Shaded}

Vamos a comparar ocho modelos para la variable S1\_dur

\begin{longtable}[]{@{}llll@{}}
\toprule
MODELO & efecto fijo & efecto aleatorio &
heterocedasticidad\tabularnewline
\midrule
\endhead
f000 & isla & constante & sin\tabularnewline
f010 & isla & isla & sin\tabularnewline
f001 & isla & constante & isla\tabularnewline
f011 & isla & isla & isla\tabularnewline
--- & --- & --- & ---\tabularnewline
f100 & isla+long & constante & sin\tabularnewline
f110 & isla+long & isla & sin\tabularnewline
f101 & isla+long & constante & isla\tabularnewline
f111 & isla+long & isla & isla\tabularnewline
\bottomrule
\end{longtable}

modelo nulo con efecto fijo de la isla

\begin{Shaded}
\begin{Highlighting}[]
\NormalTok{f000 <-}\StringTok{ }\KeywordTok{lme}\NormalTok{(S1_dur}\OperatorTok{~}\NormalTok{Region,dts,}\DataTypeTok{random=}\OperatorTok{~}\DecValTok{1}\OperatorTok{|}\NormalTok{LocCode}\OperatorTok{/}\NormalTok{IndivGroup, }\DataTypeTok{method=}\StringTok{"ML"}\NormalTok{)}
\end{Highlighting}
\end{Shaded}

nulo + efecto aleatorio de isla/continente

\begin{Shaded}
\begin{Highlighting}[]
\NormalTok{f010 <-}\StringTok{ }\KeywordTok{lme}\NormalTok{(S1_dur}\OperatorTok{~}\NormalTok{Region,dts,}
  \DataTypeTok{random=}\KeywordTok{list}\NormalTok{(}\DataTypeTok{LocCode=}\KeywordTok{pdDiag}\NormalTok{(}\OperatorTok{~}\NormalTok{Region),}\DataTypeTok{IndivGroup=}\KeywordTok{pdDiag}\NormalTok{(}\OperatorTok{~}\NormalTok{Region)), }\DataTypeTok{method=}\StringTok{"ML"}\NormalTok{)}
\end{Highlighting}
\end{Shaded}

nulo + heterocedasticidad

\begin{Shaded}
\begin{Highlighting}[]
\NormalTok{f001 <-}\StringTok{ }\KeywordTok{lme}\NormalTok{(S1_dur}\OperatorTok{~}\NormalTok{Region,dts,}\DataTypeTok{random=}\OperatorTok{~}\DecValTok{1}\OperatorTok{|}\NormalTok{LocCode}\OperatorTok{/}\NormalTok{IndivGroup,}\DataTypeTok{weights=}\KeywordTok{varIdent}\NormalTok{(}\DataTypeTok{form=}\OperatorTok{~}\DecValTok{1}\OperatorTok{|}\NormalTok{Region), }\DataTypeTok{method=}\StringTok{"ML"}\NormalTok{)}
\end{Highlighting}
\end{Shaded}

nulo + efecto aleatorio de isla/continente + heterocedasticidad

\begin{Shaded}
\begin{Highlighting}[]
\NormalTok{f011 <-}
\StringTok{ }\KeywordTok{lme}\NormalTok{(S1_dur}\OperatorTok{~}\NormalTok{Region,dts,}
   \DataTypeTok{random=}\KeywordTok{list}\NormalTok{(}\DataTypeTok{LocCode=}\KeywordTok{pdDiag}\NormalTok{(}\OperatorTok{~}\NormalTok{Region),}\DataTypeTok{IndivGroup=}\KeywordTok{pdDiag}\NormalTok{(}\OperatorTok{~}\NormalTok{Region)),}
   \DataTypeTok{weights=}\KeywordTok{varIdent}\NormalTok{(}\DataTypeTok{form=}\OperatorTok{~}\DecValTok{1}\OperatorTok{|}\NormalTok{Region), }\DataTypeTok{method=}\StringTok{"ML"}\NormalTok{)}
\end{Highlighting}
\end{Shaded}

modelo alternativo con efecto fijo de la isla y longitud

\begin{Shaded}
\begin{Highlighting}[]
\NormalTok{f100 <-}\StringTok{ }\KeywordTok{lme}\NormalTok{(S1_dur}\OperatorTok{~}\NormalTok{Region}\OperatorTok{+}\NormalTok{Long,dts,}\DataTypeTok{random=}\OperatorTok{~}\DecValTok{1}\OperatorTok{|}\NormalTok{LocCode}\OperatorTok{/}\NormalTok{IndivGroup, }\DataTypeTok{method=}\StringTok{"ML"}\NormalTok{)}
\end{Highlighting}
\end{Shaded}

alternativo + efecto aleatorio de isla/continente

\begin{Shaded}
\begin{Highlighting}[]
\NormalTok{f110 <-}\StringTok{ }\KeywordTok{lme}\NormalTok{(S1_dur}\OperatorTok{~}\NormalTok{Region}\OperatorTok{+}\NormalTok{Long,dts,}
     \DataTypeTok{random=}\KeywordTok{list}\NormalTok{(}\DataTypeTok{LocCode=}\KeywordTok{pdDiag}\NormalTok{(}\OperatorTok{~}\NormalTok{Region),}\DataTypeTok{IndivGroup=}\KeywordTok{pdDiag}\NormalTok{(}\OperatorTok{~}\NormalTok{Region)), }\DataTypeTok{method=}\StringTok{"ML"}\NormalTok{)}
\end{Highlighting}
\end{Shaded}

alternativo + heterocedasticidad

\begin{Shaded}
\begin{Highlighting}[]
\NormalTok{f101 <-}\StringTok{ }\KeywordTok{lme}\NormalTok{(S1_dur}\OperatorTok{~}\NormalTok{Region}\OperatorTok{+}\NormalTok{Long,dts,}\DataTypeTok{random=}\OperatorTok{~}\DecValTok{1}\OperatorTok{|}\NormalTok{LocCode}\OperatorTok{/}\NormalTok{IndivGroup,}\DataTypeTok{weights=}\KeywordTok{varIdent}\NormalTok{(}\DataTypeTok{form=}\OperatorTok{~}\DecValTok{1}\OperatorTok{|}\NormalTok{Region), }\DataTypeTok{method=}\StringTok{"ML"}\NormalTok{)}
\end{Highlighting}
\end{Shaded}

alternativo + efecto aleatorio de isla/continente + heterocedasticidad

\begin{Shaded}
\begin{Highlighting}[]
\NormalTok{f111 <-}
\StringTok{      }\KeywordTok{lme}\NormalTok{(S1_dur}\OperatorTok{~}\NormalTok{Region}\OperatorTok{+}\NormalTok{Long,dts,}
        \DataTypeTok{random=}\KeywordTok{list}\NormalTok{(}\DataTypeTok{LocCode=}\KeywordTok{pdDiag}\NormalTok{(}\OperatorTok{~}\NormalTok{Region),}\DataTypeTok{IndivGroup=}\KeywordTok{pdDiag}\NormalTok{(}\OperatorTok{~}\NormalTok{Region)),}
        \DataTypeTok{weights=}\KeywordTok{varIdent}\NormalTok{(}\DataTypeTok{form=}\OperatorTok{~}\DecValTok{1}\OperatorTok{|}\NormalTok{Region), }\DataTypeTok{method=}\StringTok{"ML"}\NormalTok{)}
\end{Highlighting}
\end{Shaded}

\hypertarget{resultados}{%
\subsection{Resultados}\label{resultados}}

\begin{Shaded}
\begin{Highlighting}[]
\KeywordTok{anova}\NormalTok{(f000,f010,f001,f011,}
\NormalTok{  f100,f110,f101,f111)}
\end{Highlighting}
\end{Shaded}

\begin{verbatim}
##      Model df       AIC       BIC   logLik   Test  L.Ratio p-value
## f000     1  5 -6171.918 -6145.875 3090.959                        
## f010     2  7 -6192.394 -6155.934 3103.197 1 vs 2 24.47592  <.0001
## f001     3  6 -6170.306 -6139.055 3091.153 2 vs 3 24.08761  <.0001
## f011     4  8 -6190.912 -6149.243 3103.456 3 vs 4 24.60566  <.0001
## f100     5  6 -6171.541 -6140.289 3091.771 4 vs 5 23.37118  <.0001
## f110     6  8 -6190.970 -6149.301 3103.485 5 vs 6 23.42902  <.0001
## f101     7  7 -6169.927 -6133.467 3091.963 6 vs 7 23.04318  <.0001
## f111     8  9 -6189.484 -6142.606 3103.742 7 vs 8 23.55687  <.0001
\end{verbatim}

\begin{Shaded}
\begin{Highlighting}[]
\NormalTok{mis.aics <-}\StringTok{ }\KeywordTok{AIC}\NormalTok{(f000,f010,f001,f011,}
\NormalTok{  f100,f110,f101,f111)}
\KeywordTok{cbind}\NormalTok{(mis.aics,}\DataTypeTok{delta.AIC=}\NormalTok{mis.aics[,}\DecValTok{2}\NormalTok{]}\OperatorTok{-}\KeywordTok{min}\NormalTok{(mis.aics[,}\DecValTok{2}\NormalTok{]))}
\end{Highlighting}
\end{Shaded}

\begin{verbatim}
##      df       AIC delta.AIC
## f000  5 -6171.918 20.475921
## f010  7 -6192.394  0.000000
## f001  6 -6170.306 22.087613
## f011  8 -6190.912  1.481951
## f100  6 -6171.541 20.853132
## f110  8 -6190.970  1.424114
## f101  7 -6169.927 22.467296
## f111  9 -6189.484  2.910423
\end{verbatim}

\begin{Shaded}
\begin{Highlighting}[]
\NormalTok{f010}
\end{Highlighting}
\end{Shaded}

\begin{verbatim}
## Linear mixed-effects model fit by maximum likelihood
##   Data: dts 
##   Log-likelihood: 3103.197
##   Fixed: S1_dur ~ Region 
## (Intercept)   Regionisl 
##  0.10394315  0.03296709 
## 
## Random effects:
##  Formula: ~Region | LocCode
##  Structure: Diagonal
##         (Intercept)  Regionisl
## StdDev:  0.01225407 0.03564827
## 
##  Formula: ~Region | IndivGroup %in% LocCode
##  Structure: Diagonal
##         (Intercept)  Regionisl   Residual
## StdDev:  0.01327504 0.02540965 0.02224762
## 
## Number of Observations: 1351
## Number of Groups: 
##                 LocCode IndivGroup %in% LocCode 
##                      14                      97
\end{verbatim}

\begin{Shaded}
\begin{Highlighting}[]
\KeywordTok{summary}\NormalTok{(f010)}
\end{Highlighting}
\end{Shaded}

\begin{verbatim}
## Linear mixed-effects model fit by maximum likelihood
##  Data: dts 
##         AIC       BIC   logLik
##   -6192.394 -6155.934 3103.197
## 
## Random effects:
##  Formula: ~Region | LocCode
##  Structure: Diagonal
##         (Intercept)  Regionisl
## StdDev:  0.01225407 0.03564827
## 
##  Formula: ~Region | IndivGroup %in% LocCode
##  Structure: Diagonal
##         (Intercept)  Regionisl   Residual
## StdDev:  0.01327504 0.02540965 0.02224762
## 
## Fixed effects: S1_dur ~ Region 
##                  Value   Std.Error   DF   t-value p-value
## (Intercept) 0.10394315 0.005326166 1254 19.515568  0.0000
## Regionisl   0.03296709 0.018461144   12  1.785755  0.0994
##  Correlation: 
##           (Intr)
## Regionisl -0.289
## 
## Standardized Within-Group Residuals:
##        Min         Q1        Med         Q3        Max 
## -3.6150052 -0.5554097 -0.0942740  0.3947379  5.6454153 
## 
## Number of Observations: 1351
## Number of Groups: 
##                 LocCode IndivGroup %in% LocCode 
##                      14                      97
\end{verbatim}

\begin{Shaded}
\begin{Highlighting}[]
\KeywordTok{intervals}\NormalTok{(f010)}
\end{Highlighting}
\end{Shaded}

\begin{verbatim}
## Approximate 95% confidence intervals
## 
##  Fixed effects:
##                    lower       est.      upper
## (Intercept)  0.093501706 0.10394315 0.11438459
## Regionisl   -0.007226506 0.03296709 0.07316068
## attr(,"label")
## [1] "Fixed effects:"
## 
##  Random Effects:
##   Level: LocCode 
##                       lower       est.      upper
## sd((Intercept)) 0.006287163 0.01225407 0.02388392
## sd(Regionisl)   0.016885509 0.03564827 0.07525977
##   Level: IndivGroup 
##                      lower       est.      upper
## sd((Intercept)) 0.01020718 0.01327504 0.01726497
## sd(Regionisl)   0.01870918 0.02540965 0.03450981
## 
##  Within-group standard error:
##      lower       est.      upper 
## 0.02139408 0.02224762 0.02313522
\end{verbatim}

\begin{Shaded}
\begin{Highlighting}[]
\KeywordTok{VarCorr}\NormalTok{(f010)}
\end{Highlighting}
\end{Shaded}

\begin{verbatim}
##              Variance       StdDev    
## LocCode =    pdDiag(Region)           
## (Intercept)  0.0001501621   0.01225407
## Regionisl    0.0012707994   0.03564827
## IndivGroup = pdDiag(Region)           
## (Intercept)  0.0001762266   0.01327504
## Regionisl    0.0006456502   0.02540965
## Residual     0.0004949568   0.02224762
\end{verbatim}


\end{document}
